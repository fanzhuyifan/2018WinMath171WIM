\documentclass[a4paper]{article}
\usepackage{mathrsfs}
\usepackage{tikz}
\usetikzlibrary{calc}
\usetikzlibrary{decorations.markings,arrows}
\usepackage{subcaption}
\renewcommand{\baselinestretch}{1.05}
\usepackage{amsmath,amsthm,verbatim,amssymb,amsfonts,amscd, graphicx, float,pgfplots}
\usepackage{graphics}
\usepackage{geometry}
\geometry{left=2.5cm,right=2.5cm,top=2.5cm,bottom=2.5cm}

\usepackage{thmtools}
\usepackage{thm-restate}
\usepackage{hyperref}
\usepackage[noabbrev]{cleveref}
\crefname{subsection}{subsection}{subsections}

\newtheorem{theorem}{Theorem}[section]
\newtheorem{corollary}[theorem]{Corollary}
\newtheorem{lemma}[theorem]{Lemma}
\newtheorem{proposition}[theorem]{Proposition}

\theoremstyle{definition}
\newtheorem{definition}{Definition}[section]

\theoremstyle{remark}
\newtheorem*{remark}{Remark}


\begin{document}
\title{The Infinitude of Primes}
\author{
Yifan Zhu
}
\maketitle

\begin{abstract}
  Mathematicians dating back to Euclid have known about the infinitude of primes. Since then, many different proofs have been proposed. This paper examines Hillel Furstenberg's topological proof and a series of analytical proofs that go back to Euler's proof that the sum of the reciprocals of primes diverges.
\end{abstract}

\section{Introduction}
Humans' study of math really began with natural numbers --- counting was universal. With the study of natural numbers, the field of number theory naturally emerged. Central to this field is the study of primes. Among their many interesting properties, the most well known is probably their infinitude. This result dates back thousands of years --- around 300BC, Euclid proved in \textit{Elements} that there are infinitely many primes. \cite{bib:mathHistory} Since then, mathematicians have given many different proofs. In this paper, we will examine Hillel Furstenberg's topological proof in \cref{sec:topological} and a series of analytical proofs dating back to Euler in \cref{sec:analytical}.

Hillel Furstenberg's topological proof stands out for being, well, \textbf{topological}. You might be tempted to ask, {WHAT ON EARTH DO PRIMES HAVE TO DO WITH TOPOLOGY?} As we will see later, Furstenberg made this remarkable connection (as an undergrad at Yeshiva University) by defining the evenly spaced integer topology on the set of integers, where the open sets are unions of arithmetic sequences. In light of this, his proof is more about certain properties of arithmetic sequences than about topology. \cite{bib:Furstenberg} \cite{bib:proofsFromTheBook} 

The analytical proofs center on the Euler product formula,
\[
\sum^\infty_{n=1}\frac{1}{n^s}=\prod_{p\in\mathbb{P}}\frac{1}{1-p^{-s}}
,
\]
where $\mathbb{P}$ is the set of primes.
If we forget about convergence and divergence for a moment and let $s$ take on the value $1$, then this formula establishes a relationship between the sum of the reciprocals of natural numbers and the sum of the reciprocals of primes. Using this, Euler proved (not very rigorously) that the sum of the reciprocals of the primes less than $n$ grows approximately as fast as $\ln\ln n$ as $n$ approaches infinity.

In this paper, we will see an easy proof of the infinitude of primes using a rigorous version of the Euler product formula (\cref{thm:EulerInfinitePrimes}), examine Euler's original proof (\cref{thm:EulerReciprocalOfPrimes}) and adapt it to rigorously prove that the sum of the reciprocals of primes diverge (\cref{thm:reciprocalPrimeDiverge}), and look at a proof for the lower bound of the sum (\cref{thm:sumReciprocalPrimeLnLn}).
\section{Topological Proof}
\label{sec:topological}
Let us now examine the topological proof of the infinitude of primes given by Hillel Furstenberg.
First we define the evenly spaced integer topology on the integers:
\begin{definition}
  The evenly spaced integer topology is a topology where the open sets are (possibly empty) unions of two-way infinite arithmetic sequences $S(a,b)=\left\{ a+nb:n\in\mathbb{Z} \right\}$ ($a$ and $b$ are integers). Or equivalently, a set $O\subset\mathbb{Z}$ is open if for all $a\in O$ there exists some $b\in\mathbb{Z^+}$ with $S(a,b)\subset O$.
  \label{def:esip}
\end{definition}
Before we proceed, we need to check that the evenly spaced integer topology is indeed a topology:
\begin{lemma}
  The evenly spaced integer topology is a topology on the integers $\mathbb{Z}$.
  \label{lem:esip}
\end{lemma}
\begin{proof}
  From that definition of topology, we need to check three things:
  \begin{enumerate}
    \item Both the empty set and $Z$ are open;

      Clearly, the empty set is open since in that case the definition for openness is vacuously true. $\mathbb{Z}$ is also open since for all $a\in\mathbb{Z}$, $S(a,1)=\mathbb{Z}\subset\mathbb{Z}$.
    \item Any union of open sets is open;

      For any union of open sets $O=\bigcup\limits_i O_i$, for all $a\in O$, $a\in O_i$ for some $i$. Since $O_i$ is open, there exists $b\in\mathbb{Z^+}$ with $S(a,b)\subset O_i$. So $S(a,b)\subset O$. Thus, $O$ is also open.
    \item If $O_1$ and $O_2$ are open, then $O_1\cap O_2$ is also open.

      For any $a\in O_1\cap O_2$, we can find $b_1,b_2\in\mathbb{Z^+}$ with $S(a,b_1)\subset O_1$ and $S(a,b_2)\subset O_2$. Then $S(a,b_1b_2)\subset O_1\cap O_2$. Thus $O_1\cap O_2$ is open.

  \end{enumerate}
  Hence, $\tau$ defines a topology on $\mathbb{Z}$.
\end{proof}

The \textbf{evenly spaced integer topology} has two interesting properties that we will need:
\begin{lemma}
  In the \textbf{evenly spaced integer topology} $\tau$,
  \begin{enumerate}
    \item Any non-empty open set is infinite.
    \item Any of the basis sets $S(a,b)$ is also closed.
  \end{enumerate}
  \label{lem:2prop}
\end{lemma}
\begin{remark}
  The first property implies that non-empty finite sets cannot be open, so sets with a non-empty finite complement cannot be closed.
  \label{rem:2prop}
\end{remark}
\begin{proof}
  Both of these properties easily follow from the definition:
  \begin{enumerate}
    \item If the open set $O$ is not empty, we can find $a\in O$. Thus we can find $S(a,b)\subset O$. Since $S(a,b)$ is infinite, so is $O$.
    \item Note that $\displaystyle S(a,b)=\mathbb{Z}\setminus \bigcup\limits_{i=1}^{b-1}S(a+i,b)$. So $S(a,b)$ is closed as well as open.
  \end{enumerate}
\end{proof}
With these two properties, we are ready to prove that there are infinitely many primes:
\begin{theorem}
  In the \textbf{evenly spaced integer topology}, $\mathbb{Z}\setminus\left\{ -1,+1 \right\}$ cannot be closed. But if the primes are finite, then $\mathbb{Z}\setminus\left\{ -1,+1 \right\}$ is closed. Thus, there are infinitely many primes.
  \label{thm:topPrimes}
\end{theorem}
\begin{proof}
  Since $\left\{ -1,+1 \right\}$ is finite, we know from the first property of \cref{lem:2prop} that it cannot be open. Thus, $\mathbb{Z}\setminus\left\{ -1,+1 \right\}$ cannot be closed.

  Since $-1$ and $+1$ are the only integers that are not multiples of primes,
  \[
  \mathbb{Z}\setminus\left\{ -1,+1 \right\}=\bigcup\limits_{p\in\mathbb{P}}S(0,p)
  ,
  \]
  where $\mathbb{P}$ is the set of primes.
  If the primes are finite, then from the second property of \cref{lem:2prop} we know that the right hand side is the finite union of closed sets, so it is also closed. Thus, $\mathbb{Z}\setminus\left\{ -1,+1 \right\}$ is closed. But we already know that $\mathbb{Z}\setminus\left\{ -1,+1 \right\}$ cannot be closed, so this is a contradiction.

  Therefore, there are infinitely many primes.
\end{proof}

\section{Analytical Proofs}
\label{sec:analytical}
The analytical proofs of the infinitude of primes rest on Euler's observation that 
\[
\sum^\infty_{n=1}\frac{1}{n}=\prod_{p\in\mathbb{P}}\frac{1}{1-p}
,
\]
where $\mathbb{P}$ is the set of primes.
However, since in his time Euler did not know about rigorous definitions of convergence and divergence, we need to make his observation more rigorous to suit today's needs:
\begin{lemma}
  \[
  \sum\frac{1}{m}=\prod\limits_{\substack{p\in\mathbb{P}\\p\le x}}\frac{1}{1-\frac{1}{p}}
  ,
  \]
  where the sum $\sum\frac{1}{m}$ ranges over all $m\in\mathbb{N^*}$ with only prime divisors $p\le x$ and $\mathbb{P}$ denotes the set of all primes.
  \label{lem:EulerProduct}
\end{lemma}
\begin{proof}
  First, we expand $\frac{1}{1-\frac{1}{p}}$ into infinite series:
  \begin{align*}
    \prod\limits_{\substack{p\in\mathbb{P}\\p\le x}}\frac{1}{1-\frac{1}{p}}&=\prod\limits_{\substack{p\in\mathbb{P}\\p\le x}}\sum_{k\ge0}\frac{1}{p^k}\\
    .
  \end{align*}
  Let the primes less than or equal to $x$ be $p_1,p_2,\dots,p_m$. So
  \begin{align*}
    \prod\limits_{\substack{p\in\mathbb{P}\\p\le x}}\frac{1}{1-\frac{1}{p}}&=\prod\limits_{j=1}^m\sum_{k\ge0}\frac{1}{p_j^k}\\
    &=\sum\limits_{k_1,\dots,k_m\in\mathbb{N}}\frac{1}{p_1^{k_1}p_2^{k_2}\cdots p_m^{k_m}}
    .
  \end{align*}
  Since every $m$ with only prime divisors $p\le x$ can be uniquely factorized into $p_1^{k_1}p_2^{k_2}\cdots p_m^{k_m}$, this is just equal to $\sum\frac{1}{m}$. Thus,
  \[
  \sum\frac{1}{m}=\prod\limits_{\substack{p\in\mathbb{P}\\p\le x}}\frac{1}{1-\frac{1}{p}}
  ,
  \]
  where the sum $\sum\frac{1}{m}$ ranges over all $m\in\mathbb{N^*}$ with only prime divisors $p\le x$.
\end{proof}
With this lemma, we can prove that there are infinitely many primes:
\begin{theorem}
  \[
  \ln x\le \prod\limits_{\substack{p\in\mathbb{P}\\p\le x}}\frac{1}{1-\frac{1}{p}}
  ,
  \]
  where $\mathbb{P}$ is the set of all primes. Since $\ln x$ is unbounded, there are infinitely many primes.
  \label{thm:EulerInfinitePrimes}
\end{theorem}
\begin{proof}
  Since 
  \[\int_j^{j+1}\frac{dt}{t}\le\int_j^{j+1}\frac{dt}{j}=\frac{1}{j}
  ,
  \]
  if $n\le x<n+1$, 
  \begin{align*}
    \ln x&=\int_1^x\frac{dt}{t}\\
    &\le1+\frac{1}{2}+\dots+\frac{1}{n}\\
    &\le\sum\frac{1}{m}\qquad\parbox{6cm}{where $m$ ranges over all positive integers with only prime divisors $p\le x$}\\
    &=\prod\limits_{\substack{p\in\mathbb{P}\\p\le x}}\frac{1}{1-\frac{1}{p}}\qquad\text{(\cref{lem:EulerProduct})}
    .
  \end{align*}
  If there are only $N$ primes, then there are at most $N$ terms in $\displaystyle\prod\limits_{\substack{p\in\mathbb{P}\\p\le x}}\frac{1}{1-\frac{1}{p}}$, so it must be bounded. But clearly $\ln x$ is not bounded, so there must be infinitely many primes.
\end{proof}

Not only was Euler able to prove that the primes are infinite using this, but he was able to prove a stronger result (though not rigorously by today's standards) --- that the sum of the reciprocals of primes diverge and $\displaystyle\sum_{p\le x}\frac{1}{p}\approx\ln\ln x$:

\begin{theorem}
  The sum of the reciprocals of the prime numbers,
  \[
  \frac{1}{2}+\frac{1}{3}+\frac{1}{5}+\frac{1}{7}+\frac{1}{11}+\frac{1}{13}+\dots
  \]
  is infinitely great but is infinitely times less than the sum of the harmonic series
  \[
  1+\frac{1}{2}+\frac{1}{3}+\frac{1}{4}+\frac{1}{5}+\dots
  \]
  And the sum of the former is as the logarithm of the sum of the latter.
  \cite{bib:Euler}
  \label{thm:EulerReciprocalOfPrimes}
\end{theorem}
\begin{proof}
  Euler first uses his product formula,
  \[
  \ln\left( \sum_{n=1}^\infty\frac{1}{n} \right)=\ln\left( \prod_{p\in\mathbb{P}}\frac{1}{1-p^{-1}} \right)=-\sum_{p\in\mathbb{P}}\ln\left( 1-\frac{1}{p} \right)
  .
  \]
  Then he expands $\ln(1-x)$ near $0$ to get
  \begin{align*}
    \ln\left( \sum_{n=1}^\infty\frac{1}{n} \right)&=\sum_{p\in\mathbb{P}}\left( \frac{1}{p}+\frac{1}{2p^2}+\cdots \right)\\
    &=\sum_{p\in\mathbb{P}}\frac{1}{p}+\frac{1}{2}\sum_{p\in\mathbb{P}}\frac{1}{p^2}+\frac{1}{3}\sum_{p\in\mathbb{P}}\frac{1}{p^3}+\cdots\\
    &=A+\frac{1}{2}B+\frac{1}{3}C+\frac{1}{4}D+\cdots
    ,
  \end{align*}
  where $\displaystyle A=\sum_{p\in\mathbb{P}}\frac{1}{p}$, $\displaystyle B=\sum_{p\in\mathbb{P}}\frac{1}{p^2}$, and $\displaystyle C=\sum_{p\in\mathbb{P}}\frac{1}{p^3}$\dots

  Euler further claims that $A$ is infinite, while $\frac{1}{2}B+\frac{1}{3}C+\frac{1}{4}D+\cdots$ is some fixed constant. So 
  \[
  A=\frac{1}{2}+\frac{1}{3}+\frac{1}{5}+\frac{1}{7}+\frac{1}{11}+\frac{1}{13}+\dots=\ln \left( \sum_{n=1}^\infty\frac{1}{n} \right) = \ln\ln\infty
  .
  \]
\end{proof}

The result obtained by Euler was correct, even though he obtained it using dubious methods. However, we can modify his proof to rigorously prove that the sum of the reciprocals of primes diverges:
\begin{theorem}
  If the sum of the reciprocals of primes converge, then so does the the sum of the reciprocals of positive integers. Since the sum of the reciprocals of the positive integers diverges, so does the sum of the reciprocals of primes.
  \label{thm:reciprocalPrimeDiverge}
\end{theorem}
\begin{proof}
  Suppose that the sum of the reciprocals of primes converges:
  \[
  \sum_{p\in\mathbb{P}}\frac{1}{p}=S
  .
  \]
  So
  \begin{align*}
    \sum_{k=1}^\infty\frac{1}{k}\sum_{p\in\mathbb{P}}\frac{1}{p^k}&\le S + \sum_{k=2}^\infty\frac{1}{k}\sum_{n\ge2}\frac{1}{n^k}\\
    &\le S+\sum_{k=2}^\infty\frac{1}{k}\int_1^\infty\frac{dt}{t^k}\\
    &=S+\sum_{k=2}^\infty\frac{1}{k(k-1)}\\
    &=S+\sum_{k=2}^\infty\left( \frac{1}{k-1}-\frac{1}{k} \right)\\
    &=S+1
    .
  \end{align*}
  Since all the terms are positive, 
  \[
  \displaystyle\sum_{k=1}^\infty\frac{1}{k}\sum_{p\in\mathbb{P}}\frac{1}{p^k}=\sum_{p\in\mathbb{P}}\sum_{k=1}^\infty\frac{1}{kp^k}
  \]
  also converges. But for $|x|<1$, 
  \[
  -\ln(1-x)=\sum_{k=1}^\infty\frac{x^k}{k}
  ,
  \]
  so
  \[
  \sum_{p\in\mathbb{P}}\sum_{k=1}^\infty\frac{1}{kp^k}=\sum_{p\in\mathbb{P}}\left( -\ln\left( 1-\frac{1}{p} \right) \right)=\ln\left( \prod_{p\in\mathbb{P}}\frac{1}{1-p^{-1}} \right)
  .
  \]
  Thus, $\displaystyle\prod_{p\in\mathbb{P}}\frac{1}{1-p^{-1}}$ also converges. But
  \begin{align*}
    \prod_{p\in\mathbb{P}}\frac{1}{1-p^{-1}}&=\lim_{x\to\infty}\prod\limits_{\substack{p\in\mathbb{P}\\p\le x}}\frac{1}{1-\frac{1}{p}}\\
    &=\lim_{x\to\infty}\left( \sum\frac{1}{m}\qquad\parbox{6cm}{where $m$ ranges over all positive integers with only prime divisors $p\le x$} \right)\\
    &=\sum_{n=1}^\infty\frac{1}{n}
  \end{align*}
  So $\displaystyle\sum_{n=1}^\infty\frac{1}{n}$ must also converge, which is a contradiction.

  Therefore, $\displaystyle\sum_{p\in\mathbb{P}}\frac{1}{p}$ diverges.
\end{proof}
A full proof of Euler's result is beyond the scope of this paper, so we aim for less ambitious goals. We only prove a lower bound for the sum of the reciprocals of primes less than or equal to $n$ in the theorem bellow:

\begin{theorem}
  \[
  \sum_{\substack{p\in\mathbb{P}\\p\le n}}\frac{1}{p}>\ln\ln(n+1)-\ln\frac{5}{3}
  .
  \]
  \label{thm:sumReciprocalPrimeLnLn}
\end{theorem}
\begin{remark}
  This means that the sum of the reciprocals of primes less than $n$ grows at least as fast as $\ln\ln n$.
  \label{rem:sumReciprocalPrimeLnLn}
\end{remark}
\begin{proof}
  First, since all positive integers can be uniquely expressed as the product of a square-free integer and a sqaure,
  \[
  \sum_{j=1}^n\frac{1}{j}\le\prod_{\substack{p\in\mathbb{P}\\p\le n}}\left( 1+\frac{1}{p} \right)\sum_{k=1}^n\frac{1}{k^2}
  ,
  \]
  where $\displaystyle\prod_{\substack{p\in\mathbb{P}\\p\le n}}\left( 1+\frac{1}{p} \right)$ gives us the reciprocals of all square-free integers and $\displaystyle\sum_{k=1}^n\frac{1}{k^2}$ gives us all squares less than or equal to $n$.
  We estimate the left side with
  \begin{align*}
    \ln(n+1)&=\int_1^{n+1}\frac{dt}{t}\\
    &=\sum_{j=1}^n\int_j^{j+1}\frac{dt}{t}\\
    &\le\sum_{j=1}^n\int_j^{j+1}\frac{dt}{j}\\
    &=\sum_{j=1}^n\frac{1}{j}
    .
  \end{align*}
  And we estimate the right side with
  \begin{align*}
    \sum_{k=1}^n\frac{1}{k^2}&\le1+\sum_{k=2}^n\left( \frac{1}{k-\frac{1}{2}}-\frac{1}{k+\frac{1}{2}} \right)\\
    &=1+\frac{2}{3}-\frac{1}{n+\frac{1}{2}}\\
    &<\frac{5}{3}
    .
  \end{align*}
  Combine these two estimates and we get
  \[
  \ln(n+1)\le\sum_{j=1}^n\frac{1}{j}\le\prod_{\substack{p\in\mathbb{P}\\p\le n}}\left( 1+\frac{1}{p} \right)\sum_{k=1}^n\frac{1}{k^2}<\frac{5}{3}\prod_{\substack{p\in\mathbb{P}\\p\le n}}\left( 1+\frac{1}{p} \right)
  .
  \]
  But for all $x$, $1+x\le e^{x}$, so
  \[
  \prod_{\substack{p\in\mathbb{P}\\p\le n}}\left( 1+\frac{1}{p} \right)\le \exp\left( \sum_{\substack{p\in\mathbb{P}\\p\le n}}\frac{1}{p} \right)
  .
  \]
  Thus, 
  \[
  \ln\ln(n+1)-\ln\frac{5}{3}<\sum_{\substack{p\in\mathbb{P}\\p\le n}}\frac{1}{p}
  .
  \]
\end{proof}

\begin{thebibliography}{12}
  \bibitem{bib:mathHistory}
    \textit{Mathematics and Its History}, Stillwell, Springer (2010)
  \bibitem{bib:Furstenberg}
    \textit{On the Infinitude of primes}, Furstenberg, American Mathematical Monthly (1955) 62 (5): 353
  \bibitem{bib:proofsFromTheBook}
    \textit{Proofs from The Book}, Aigner, Ziegler, Springer-Verlag (1998)
  \bibitem{bib:Euler}
    \textit{Variae observationes circa series infinitas (E72)}, Euler, Commentarii academiae scientiarum
    Petropolitanae 9 (1744), 160–188
\end{thebibliography}
\end{document}
