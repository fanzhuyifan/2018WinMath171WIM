\documentclass[a4paper]{article}
\usepackage{mathrsfs}
\usepackage{tikz}
\usetikzlibrary{calc}
\usetikzlibrary{decorations.markings,arrows}
\usepackage{subcaption}
\renewcommand{\baselinestretch}{1.05}
\usepackage{amsmath,amsthm,verbatim,amssymb,amsfonts,amscd, graphicx, float,pgfplots}
\usepackage{graphics}
\usepackage{geometry}
\geometry{left=2.5cm,right=2.5cm,top=2.5cm,bottom=2.5cm}

\usepackage{thmtools}
\usepackage{thm-restate}
\usepackage{hyperref}
\usepackage[noabbrev]{cleveref}
\Crefname{subsection}{subsection}{subsections}

\newtheorem{theorem}{Theorem}[section]
\newtheorem{corollary}{Corollary}[theorem]
\newtheorem{lemma}[theorem]{Lemma}
\newtheorem{proposition}[theorem]{Proposition}

\theoremstyle{definition}
\newtheorem{definition}{Definition}[section]

\theoremstyle{remark}
\newtheorem*{remark}{Remark}


\begin{document}
\title{The Infinitude of Primes}
\author{
Yifan Zhu
}
\maketitle

\begin{abstract}
  Mathematicians have known about the infinitude of primes from Euclid's time. Since then, many different proofs have been proposed. This paper examines Hillel Furstenberg's topological proof and a series of analytical proofs that go back to Euler's proof that the sum of the reciprocals of primes diverges.
\end{abstract}

\section{Introduction}
The history of mathematics began with natural numbers, from which number theory naturally emerged. Central to this field is the study of primes. Among their many interesting properties, the most well known is probably their infinitude. This result dates back thousands of years --- around 300BC, Euclid proved in \textit{Elements} that there are infinitely many primes\cite{bib:mathHistory}. Since then, mathematicians have given many different proofs. In this paper, we examine Hillel Furstenberg's topological proof in \Cref{sec:topological} and a series of analytical proofs dating back to Euler in \Cref{sec:analytical}.

Hillel Furstenberg's topological proof stands out for being, well, \textbf{topological}. As detailed in \Cref{sec:topological}, Furstenberg made the remarkable connection between topology and primes (as an undergraduate student at Yeshiva University) by defining the evenly spaced integer topology on the set of integers, where the open sets are unions of arithmetic sequences. In light of this, his proof is more about certain properties of arithmetic sequences than about topology. \cite{bib:Furstenberg} \cite{bib:proofsFromTheBook} 

The analytical proofs center on the Euler product formula,
\[
\sum^\infty_{n=1}\frac{1}{n^s}=\prod_{p\in\mathbb{P}}\frac{1}{1-p^{-s}}
,
\]
where $\mathbb{P}$ is the set of primes.
If we forget about convergence and divergence and let $s$ take on the value $1$, then this formula establishes a relationship between the sum of the reciprocals of natural numbers and the sum of the reciprocals of primes. Using this, Euler proved (not very rigorously) that the sum of the reciprocals of the primes less than $n$ grows approximately as fast as $\ln\ln n$.

In this paper, we will see an easy proof of the infinitude of primes using a rigorous version of the Euler product formula (\Cref{thm:EulerInfinitePrimes} and \Cref{cor:EulerInfinitePrimes}), examine Euler's original proof (\Cref{thm:EulerReciprocalOfPrimes}) and adapt it to rigorously prove that the sum of the reciprocals of primes diverge (\Cref{thm:reciprocalPrimeDiverge} and \Cref{cor:reciprocalPrimeDiverge}), and look at a proof for the lower bound of the sum (\Cref{thm:sumReciprocalPrimeLnLn}).
\section{Topological Proof}
\label{sec:topological}
Let us now examine the topological proof of the infinitude of primes given by Hillel Furstenberg.
First we define the \textbf{evenly spaced integer topology} on the integers:
\begin{definition}
  The \textbf{evenly spaced integer topology} is a topology where a set $O\subset\mathbb{Z}$ is open if for all $a\in O$ there exists some $b\in\mathbb{Z^+}$ with $S(a,b)\subset O$, where $S(a,b)$ is the two-way infinite arithmetic sequences $\left\{ a+nb:n\in\mathbb{Z} \right\}$.
  \label{def:esip}
\end{definition}
Before we proceed, we need to check that the evenly spaced integer topology is indeed a topology:
\begin{lemma}
  The evenly spaced integer topology is a topology on the integers $\mathbb{Z}$.
  \label{lem:esip}
\end{lemma}
\begin{proof}
  From that definition of topology, we need to check three things:
  \begin{enumerate}
    \item Both the empty set and $Z$ are open;

      Clearly, the empty set is open since in that case the definition for openness is vacuously true. $\mathbb{Z}$ is also open since for all $a\in\mathbb{Z}$, $S(a,1)=\mathbb{Z}\subset\mathbb{Z}$.
    \item Any union of open sets is open;

      For any union of open sets $O=\bigcup\limits_i O_i$, for all $a\in O$, $a\in O_i$ for some $i$. Since $O_i$ is open, there exists $b\in\mathbb{Z^+}$ with $S(a,b)\subset O_i$. So $S(a,b)\subset O$. Thus, $O$ is also open.
    \item If $O_1$ and $O_2$ are open, then $O_1\cap O_2$ is also open.

      For any $a\in O_1\cap O_2$, we can find $b_1,b_2\in\mathbb{Z^+}$ with $S(a,b_1)\subset O_1$ and $S(a,b_2)\subset O_2$. Then $S(a,b_1b_2)\subset O_1\cap O_2$. Thus $O_1\cap O_2$ is open.

  \end{enumerate}
  Hence, the evenly spaced integer topology is indeed a topology on $\mathbb{Z}$.
\end{proof}
Hence forward in this section, unless specified otherwise, all open or closed sets refer to open or closed sets in the evenly spaced integer topology.
The evenly spaced integer topology has two interesting properties that we will need:
\begin{lemma}
  In the evenly spaced integer topology,
  \begin{enumerate}
    \item Any non-empty open set is infinite.
    \item Any of the basis sets $S(a,b)$ is also closed.
  \end{enumerate}
  \label{lem:2prop}
\end{lemma}
\begin{remark}
  The first property implies that non-empty finite sets cannot be open, so sets with a non-empty finite complement cannot be closed.
  \label{rem:2prop}
\end{remark}
\begin{proof}
  Both of these properties easily follow from the definition:
  \begin{enumerate}
    \item If the open set $O$ is not empty, we can find $a\in O$. Since $O$ is open, there exists $b\in\mathbb{Z}^+$ such that $S(a,b)\subset O$. Since $S(a,b)$ is infinite, so is $O$.
    \item Note that $\displaystyle S(a,b)=\mathbb{Z}\setminus \bigcup\limits_{i=1}^{b-1}S(a+i,b)$. So $S(a,b)$ is closed as well as open.
  \end{enumerate}
\end{proof}
With these two properties, we can prove that there are infinitely many primes:
\begin{theorem}
  If the primes are finite, then the set $\mathbb{Z}\setminus\left\{ -1,+1 \right\}$ is closed in the evenly spaced integer topology.
  \label{thm:topPrimes}
\end{theorem}
\begin{proof}
  Since $-1$ and $+1$ are the only integers that are not multiples of primes,
  \[
  \mathbb{Z}\setminus\left\{ -1,+1 \right\}=\bigcup\limits_{p\in\mathbb{P}}S(0,p)
  ,
  \]
  where $\mathbb{P}$ is the set of primes.
  If the primes are finite, then from the second property of \Cref{lem:2prop} we know that the right hand side is the finite union of closed sets, so it is also closed. Thus, $\mathbb{Z}\setminus\left\{ -1,+1 \right\}$ is closed. 
\end{proof}
\begin{corollary}
  There are infinitely many primes.
  \label{cor:topPrimes}
\end{corollary}
\begin{proof}
  Since $\left\{ -1,+1 \right\}$ is finite, we know from the first property of \Cref{lem:2prop} that it cannot be open in the evenly spaced integer topology. Thus, $\mathbb{Z}\setminus\left\{ -1,+1 \right\}$ cannot be closed. However, \Cref{thm:topPrimes} tells us that if the primes were finite,  $\mathbb{Z}\setminus\left\{ -1,+1 \right\}$ would be closed, which would be a contradiction. Therefore, there are infinitely many primes.
\end{proof}

\section{Analytical Proofs}
\label{sec:analytical}
The analytical proofs of the infinitude of primes rest on Euler's observation that 
\[
\sum^\infty_{n=1}\frac{1}{n}=\prod_{p\in\mathbb{P}}\frac{1}{1-\frac{1}{p}}
,
\]
where $\mathbb{P}$ is the set of primes. However, this observation is not rigorous in the sense that both sides are divergent. A rigorous version is provided below:
\begin{lemma}
  For all $x>0$, the sum of reciprocals of positive integers with only prime divisors less than or equal to $x$, $\displaystyle \sum_{m\in I_x}\frac{1}{m}$, is equal to $\displaystyle \prod\limits_{\substack{p\in\mathbb{P}\\p\le x}}\frac{1}{1-\frac{1}{p}}$:
  \[
  \sum_{m\in I_x}\frac{1}{m}=\prod\limits_{\substack{p\in\mathbb{P}\\p\le x}}\frac{1}{1-\frac{1}{p}}
  ,
  \]
  where $I_x$ is the set of integers $m\in\mathbb{N^*}$ with only prime divisors $p\le x$ and $\mathbb{P}$ is the set of all primes.
  \label{lem:EulerProduct}
\end{lemma}
\begin{proof}
  First, we expand $\frac{1}{1-\frac{1}{p}}$ into infinite series:
  \[
  \prod\limits_{\substack{p\in\mathbb{P}\\p\le x}}\frac{1}{1-\frac{1}{p}}=\prod\limits_{\substack{p\in\mathbb{P}\\p\le x}}\sum_{k\ge0}\frac{1}{p^k}
  .
  \]
  Let the primes less than or equal to $x$ be $p_1,p_2,\dots,p_n$. So
  \[
  \prod\limits_{\substack{p\in\mathbb{P}\\p\le x}}\frac{1}{1-\frac{1}{p}}=\prod\limits_{j=1}^n\sum_{k\ge0}\frac{1}{p_j^k}=\sum\limits_{k_1,\dots,k_n\in\mathbb{N}}\frac{1}{p_1^{k_1}p_2^{k_2}\cdots p_n^{k_n}}
  .
  \]
  Since every $m$ with only prime divisors $p\le x$ can be uniquely factorized into $p_1^{k_1}p_2^{k_2}\cdots p_n^{k_n}$, the sum on the right is equal to $\displaystyle\sum_{m\in I_x}\frac{1}{m}$. Thus,
  \[
  \sum_{m\in I_x}\frac{1}{m}=\prod\limits_{\substack{p\in\mathbb{P}\\p\le x}}\frac{1}{1-\frac{1}{p}}
  .
  \]
\end{proof}
With this lemma, we can prove that there are infinitely many primes:
\begin{theorem}
  For all $x>0$,
  \[
  \ln x\le \prod\limits_{\substack{p\in\mathbb{P}\\p\le x}}\frac{1}{1-\frac{1}{p}}
  ,
  \]
  where $\mathbb{P}$ is the set of all primes. 
  \label{thm:EulerInfinitePrimes}
\end{theorem}
\begin{proof}
  First note that
  \[\int_j^{j+1}\frac{dt}{t}\le\int_j^{j+1}\frac{dt}{j}=\frac{1}{j}
  .
  \]
  Thus, if $n\le x<n+1$, 
  \[
  \ln x=\int_1^x\frac{dt}{t}\le\int_1^{n+1}\frac{dt}{t}\le1+\frac{1}{2}+\dots+\frac{1}{n}
  .
  \]
  Since $I_x$, the set of all positive integers $m$ with only prime divisors $p\le x$, contains all integers from $1$ to $n$,
  \[
  \ln x\le1+\frac{1}{2}+\dots+\frac{1}{n}\le\sum_{m\in I_x}\frac{1}{m}
  .
  \]
  \Cref{lem:EulerProduct} tells us that 
  \[\sum_{m\in I_x}\frac{1}{m}=\prod\limits_{\substack{p\in\mathbb{P}\\p\le x}}\frac{1}{1-\frac{1}{p}}
  ,
  \]
  so
  \[
  \ln x\le \prod\limits_{\substack{p\in\mathbb{P}\\p\le x}}\frac{1}{1-\frac{1}{p}}
  .
  \]
\end{proof}
\begin{corollary}
  There are infinitely many primes.
  \label{cor:EulerInfinitePrimes}
\end{corollary}
\begin{proof}
  If there are only $N$ primes, then there are at most $N$ terms in $\displaystyle\prod\limits_{\substack{p\in\mathbb{P}\\p\le x}}\frac{1}{1-\frac{1}{p}}$, so it must be bounded, which implies that $\ln x$ is also bounde is also boundedd. But clearly $\ln x$ is not bounded, so there must be infinitely many primes.
\end{proof}

Not only was Euler able to prove that the primes are infinite using this, but he was able to prove a stronger result (though not rigorously by today's standards) --- that the sum of the reciprocals of primes diverge and that $\displaystyle\sum_{\substack{p\le x\\p\in\mathbb{P}}}\frac{1}{p}\approx\ln\ln x$:

\begin{theorem}
  The sum of the reciprocals of the prime numbers,
  \[
  \frac{1}{2}+\frac{1}{3}+\frac{1}{5}+\frac{1}{7}+\frac{1}{11}+\frac{1}{13}+\dots
  \]
  is infinitely great but is infinitely times less than the sum of the harmonic series
  \[
  1+\frac{1}{2}+\frac{1}{3}+\frac{1}{4}+\frac{1}{5}+\dots
  \]
  And the sum of the former is as the logarithm of the sum of the latter.
  \label{thm:EulerReciprocalOfPrimes}
\end{theorem}
\begin{remark}
  The archaic wording of this theorem is taken directly from a English translation of Euler's original paper\cite{bib:Euler}. Both the original and an English translation can be found on \textit{The Euler Archive} (http://eulerarchive.maa.org/pages/E072.html).
\end{remark}
\begin{proof}
  Euler first uses his product formula,
  \[
  \ln\left( \sum_{n=1}^\infty\frac{1}{n} \right)=\ln\left( \prod_{p\in\mathbb{P}}\frac{1}{1-p^{-1}} \right)=-\sum_{p\in\mathbb{P}}\ln\left( 1-\frac{1}{p} \right)
  .
  \]
  Then he expands $\ln(1-x)$ near $0$ to get
  \begin{align*}
    \ln\left( \sum_{n=1}^\infty\frac{1}{n} \right)&=\sum_{p\in\mathbb{P}}\left( \frac{1}{p}+\frac{1}{2p^2}+\cdots \right)\\
    &=\sum_{p\in\mathbb{P}}\frac{1}{p}+\frac{1}{2}\sum_{p\in\mathbb{P}}\frac{1}{p^2}+\frac{1}{3}\sum_{p\in\mathbb{P}}\frac{1}{p^3}+\cdots\\
    &=A+\frac{1}{2}B+\frac{1}{3}C+\frac{1}{4}D+\cdots
    ,
  \end{align*}
  where $\displaystyle A=\sum_{p\in\mathbb{P}}\frac{1}{p}$, $\displaystyle B=\sum_{p\in\mathbb{P}}\frac{1}{p^2}$, and $\displaystyle C=\sum_{p\in\mathbb{P}}\frac{1}{p^3}$\dots

  Euler further claims that $A$ is infinite, while $\frac{1}{2}B+\frac{1}{3}C+\frac{1}{4}D+\cdots$ is some fixed constant. So 
  \[
  A=\frac{1}{2}+\frac{1}{3}+\frac{1}{5}+\frac{1}{7}+\frac{1}{11}+\frac{1}{13}+\dots=\ln \left( \sum_{n=1}^\infty\frac{1}{n} \right) = \ln\ln\infty
  .
  \]
\end{proof}

The result obtained by Euler was correct, even though he obtained it using dubious methods. However, we can modify his proof to rigorously prove that the sum of the reciprocals of primes diverges:
\begin{theorem}
  If the sum of the reciprocals of primes converges, then so does the the sum of the reciprocals of positive integers.
  \label{thm:reciprocalPrimeDiverge}
\end{theorem}
\begin{proof}
  Suppose that the sum of the reciprocals of primes converges to $S$:
  \[
  \sum_{p\in\mathbb{P}}\frac{1}{p}=S
  .
  \]
  Thus,
  \begin{align*}
    \sum_{k=1}^\infty\frac{1}{k}\sum_{p\in\mathbb{P}}\frac{1}{p^k}&\le S + \sum_{k=2}^\infty\frac{1}{k}\sum_{n\ge2}\frac{1}{n^k}\\
    &\le S+\sum_{k=2}^\infty\frac{1}{k}\int_1^\infty\frac{dt}{t^k}\\
    &=S+\sum_{k=2}^\infty\frac{1}{k(k-1)}\\
    &=S+\sum_{k=2}^\infty\left( \frac{1}{k-1}-\frac{1}{k} \right)\\
    &=S+1
    .
  \end{align*}
  Since all terms are positive, we can rearrange the terms and see that
  \[
  \displaystyle\sum_{k=1}^\infty\frac{1}{k}\sum_{p\in\mathbb{P}}\frac{1}{p^k}=\sum_{p\in\mathbb{P}}\sum_{k=1}^\infty\frac{1}{kp^k}
  \]
  also converges. But for $|x|<1$, 
  \[
  -\ln(1-x)=\sum_{k=1}^\infty\frac{x^k}{k}
  ,
  \]
  so
  \[
  \sum_{p\in\mathbb{P}}\sum_{k=1}^\infty\frac{1}{kp^k}=\sum_{p\in\mathbb{P}}\left( -\ln\left( 1-\frac{1}{p} \right) \right)=\ln\left( \prod_{p\in\mathbb{P}}\frac{1}{1-p^{-1}} \right)
  .
  \]
  Thus, $\displaystyle\prod_{p\in\mathbb{P}}\frac{1}{1-p^{-1}}$ also converges. But
  \begin{align*}
    \prod_{p\in\mathbb{P}}\frac{1}{1-p^{-1}}&=\lim_{x\to\infty}\prod\limits_{\substack{p\in\mathbb{P}\\p\le x}}\frac{1}{1-\frac{1}{p}}\\
    &=\lim_{x\to\infty}\left( \sum_{m\in I_x}\frac{1}{m} \right)\\
    &=\sum_{n=1}^\infty\frac{1}{n}
    ,
  \end{align*}
  where $I_x$ is the set of all positive integers $m$ with only prime divisors $p\le x$.
  So $\displaystyle\sum_{n=1}^\infty\frac{1}{n}$ also converges.

\end{proof}
\begin{corollary}
  There are infinitely many primes.
  \label{cor:reciprocalPrimeDiverge}
\end{corollary}
\begin{proof}
  From \Cref{thm:reciprocalPrimeDiverge}, we know that if the sum of the reciprocals of primes converges, then the sum of the reciprocals of positive integers converges. But we know that the sum of the reciprocals of positive integers diverges. So the sum of the reciprocals of primes diverges. Therefore, there are infinitely many primes.
\end{proof}
A full proof of Euler's result is beyond the scope of this paper, so we aim for less ambitious goals. We only prove a lower bound for the sum of the reciprocals of primes less than or equal to $n$ in the theorem bellow:

\begin{theorem}
  The sum of the reciprocals of primes less than or equal to $n$ is strictly bounded below by $\ln\ln(n+1)-\ln\frac{5}{3}$:
  \[
  \sum_{\substack{p\in\mathbb{P}\\p\le n}}\frac{1}{p}>\ln\ln(n+1)-\ln\frac{5}{3}
  .
  \]
  \label{thm:sumReciprocalPrimeLnLn}
\end{theorem}
\begin{remark}
  This means that the sum of the reciprocals of primes less than $n$ grows at least as fast as $\ln\ln n$.
  \label{rem:sumReciprocalPrimeLnLn}
\end{remark}
\begin{proof}
  First, since all positive integers can be uniquely expressed as the product of a square-free integer and a sqaure, we must have
  \[
  \sum_{j=1}^n\frac{1}{j}\le\prod_{\substack{p\in\mathbb{P}\\p\le n}}\left( 1+\frac{1}{p} \right)\sum_{k=1}^n\frac{1}{k^2}
  ,
  \]
  where $\displaystyle\prod_{\substack{p\in\mathbb{P}\\p\le n}}\left( 1+\frac{1}{p} \right)$ gives us the reciprocals of all square-free integers less than or equal to $n$ and $\displaystyle\sum_{k=1}^n\frac{1}{k^2}$ gives us all squares less than or equal to $n$.
  We estimate the left hand side with
  \begin{align*}
    \ln(n+1)&=\int_1^{n+1}\frac{dt}{t}\\
    &=\sum_{j=1}^n\int_j^{j+1}\frac{dt}{t}\\
    &\le\sum_{j=1}^n\int_j^{j+1}\frac{dt}{j}\\
    &=\sum_{j=1}^n\frac{1}{j}
    .
  \end{align*}
  And we estimate the right hand side with
  \begin{align*}
    \sum_{k=1}^n\frac{1}{k^2}&\le1+\sum_{k=2}^n\left( \frac{1}{k-\frac{1}{2}}-\frac{1}{k+\frac{1}{2}} \right)\\
    &=1+\frac{2}{3}-\frac{1}{n+\frac{1}{2}}\\
    &<\frac{5}{3}
    .
  \end{align*}
  Combine these two estimates and we get
  \[
  \ln(n+1)\le\sum_{j=1}^n\frac{1}{j}\le\prod_{\substack{p\in\mathbb{P}\\p\le n}}\left( 1+\frac{1}{p} \right)\sum_{k=1}^n\frac{1}{k^2}<\frac{5}{3}\prod_{\substack{p\in\mathbb{P}\\p\le n}}\left( 1+\frac{1}{p} \right)
  .
  \]
  But for all $x$, $1+x\le e^{x}$, so
  \[
  \prod_{\substack{p\in\mathbb{P}\\p\le n}}\left( 1+\frac{1}{p} \right)\le \exp\left( \sum_{\substack{p\in\mathbb{P}\\p\le n}}\frac{1}{p} \right)
  .
  \]
  Thus,
  \[
  \ln(n+1)\le \frac{5}{3} \exp \left( \sum_{\substack{p\in\mathbb{P}\\p\le n}}\frac{1}{p} \right)
  .
  \]
  Take the logarithm of both sides, rearrage the terms, and we get
  \[
  \ln\ln(n+1)-\ln\frac{5}{3}<\sum_{\substack{p\in\mathbb{P}\\p\le n}}\frac{1}{p}
  .
  \]
\end{proof}

\begin{thebibliography}{12}
  \bibitem{bib:mathHistory}
    \textit{Mathematics and Its History}, Stillwell, Springer (2010)
  \bibitem{bib:Furstenberg}
    \textit{On the Infinitude of primes}, Furstenberg, American Mathematical Monthly (1955) 62 (5): 353
  \bibitem{bib:proofsFromTheBook}
    \textit{Proofs from The Book}, Aigner, Ziegler, Springer-Verlag (1998)
  \bibitem{bib:Euler}
    \textit{Variae observationes circa series infinitas (E72)}, Euler, Commentarii academiae scientiarum
    Petropolitanae 9 (1744), 160–188
\end{thebibliography}
\end{document}
