\documentclass[a4paper]{article}
\usepackage{mathrsfs}
\usepackage{tikz}
\usetikzlibrary{calc}
\usetikzlibrary{decorations.markings,arrows}
\usepackage{subcaption}
\renewcommand{\baselinestretch}{1.05}
\usepackage{amsmath,amsthm,verbatim,amssymb,amsfonts,amscd, graphicx, float,pgfplots}
\usepackage{graphics}
\usepackage{geometry}
\geometry{left=2.5cm,right=2.5cm,top=2.5cm,bottom=2.5cm}

\usepackage{thmtools}
\usepackage{thm-restate}
\usepackage{hyperref}
\usepackage[noabbrev]{cleveref}
\crefname{subsection}{subsection}{subsections}

\newtheorem{theorem}{Theorem}[section]
\newtheorem{corollary}[theorem]{Corollary}
\newtheorem{lemma}[theorem]{Lemma}
\newtheorem{proposition}[theorem]{Proposition}

\theoremstyle{definition}
\newtheorem{definition}{Definition}[section]

\theoremstyle{remark}
\newtheorem*{remark}{Remark}


\begin{document}
\title{The Infinitude of Primes}
\author{
Yifan Zhu
}
\maketitle

\begin{abstract}
  Abstract here.
\end{abstract}<++>

\section{Introduction}
Humans' study of math really began with natural numbers --- the need to count was universal. With the study of natural numbers, the field of number theory naturally emerged. Central to this field is the study of primes. Among their many interesting properties, the most well known is probably their infinitude. This result dates back thousands of years --- around 300BC, Euclid proved in \textit{Elements} that there are infinitely many primes. \cite{bib:mathHistory} Since then, mathematicians have given many different proofs. In this paper, we will examine Hillel Furstenberg's topological proof and a series of analytical proofs dating back to Euler.

Hillel Furstenberg's topological proof stands out for being, well, \textbf{topological}. You might be tempted to ask, {WHAT ON EARTH DO PRIMES HAVE TO DO WITH TOPOLOGY?} As we will see later, Furstenberg made this remarkable connection (as an undergrad at Yeshiva University) by defining the \textbf{evenly spaced integer topology} on the set of integers, where the open sets are unions of arithmetic sequences. In light of this, his proof is more about certain properties of arithmetic sequences than about topology. \cite{bib:proofsFromTheBook} \cite{bib:Furstenberg}

The analytical proofs center on the Euler product formula,
\[
\sum^\infty_{n=1}\frac{1}{n^s}=\prod_{p\text{ prime}}\frac{1}{1-p^{-s}}
.
\]
If we forget about convergence and divergence for a moment and let $s$ take on the value $1$, then this formula establishes a relationship between the sum of the reciprocals of natural numbers and the sum of the reciprocals of primes. Using this, Euler proved (not very rigorously) that the sum of the reciprocals of the primes less than $n$ grows approximately as fast as $\ln\ln n$ as $n$ approaches infinity.

<To be completed>

\section{Topological Proof}
Let us now examine the topological proof of the infinitude of primes given by Hillel Furstenberg.
First we define the \textbf{evenly spaced integer topology} on the integers:
\begin{lemma}
  Call a set $O\subset\mathbb{Z}$ open if for all $a\in O$ there exists some $b\in\mathbb{Z^+}$ with $S(a,b)\subset O$, where $S(a,b)$ is the two-way infinite arithmetic sequence $\left\{ a+nb:n\in\mathbb{Z} \right\}$. Then the collection of open sets $\tau$ forms a topology on the integers $Z$.
  \label{lem:esip}
\end{lemma}
\begin{remark}
  This topology is called the \textbf{evenly spaced integer topology}.
  \label{rem:esip}
\end{remark}
\begin{proof}
  From that definition of topology, we need to check that
  \begin{enumerate}
    \item Both the empty set and $Z$ are open;

      Clearly, the empty set is open since the definition for openness is vacuously true. $\mathbb{Z}$ is also open since for all $a\in\mathbb{Z}$, $S(a,1)=\mathbb{Z}\subset\mathbb{Z}$.
    \item Any union of open sets is open;

      For any union of open sets $O=\bigcup\limits_i O_i$, for all $a\in O$, $a\in O_i$ for some $i$. Since $O_i$ is open, there exists $b\in\mathbb{Z^+}$ with $S(a,b)\subset O_i$. So $S(a,b)\subset O$. Thus, $O$ is also open.
    \item If $O_1$ and $O_2$ are open, then $O_1\cap O_2$ is also open.

      For any $a\in O_1\cap O_2$, we can find $b_1,b_2\in\mathbb{Z^+}$ with $S(a,b_1)\subset O_1$ and $S(a,b_2)\subset O_2$. Then $S(a,b_1b_2)\subset O_1\cap O_2$. Thus $O_1\cap O_2$ is open.

  \end{enumerate}
  Hence, $\tau$ defines a topology on $\mathbb{Z}$.
\end{proof}

The \textbf{evenly spaced integer topology} has two interesting properties that we will need:
\begin{lemma}
  In the \textbf{evenly spaced integer topology} $\tau$,
  \begin{enumerate}
    \item Any non-empty open set is infinite.
    \item Any of the basis sets $S(a,b)$ is also closed.
  \end{enumerate}
  \label{lem:2prop}
\end{lemma}
\begin{proof}
  Both of these properties easily follow from the definition:
  \begin{enumerate}
    \item If the open set $O$ is not empty, we can find $a\in O$. Thus we can find $S(a,b)\subset O$, so $O$ is infinite.
    \item Note that $\displaystyle S(a,b)=\mathbb{Z}\setminus \bigcup\limits_{i=1}^{b-1}S(a+i,b)$, so $S(a,b)$ is closed as well as open.
  \end{enumerate}
\end{proof}
With these two properties, we are ready to prove that there are infinitely many primes:
\begin{theorem}
  In the \textbf{evenly spaced integer topology}, $\mathbb{Z}\setminus\left\{ -1,+1 \right\}$ cannot be closed. But if the primes are finite, then $\mathbb{Z}\setminus\left\{ -1,+1 \right\}$ is closed. Thus, there are infinitely many primes.
  \label{thm:topPrimes}
\end{theorem}
\begin{proof}
  Since $\left\{ -1,+1 \right\}$ is finite, we know from the first property of \cref{lem:2prop} that it cannot be open. Thus, $\mathbb{Z}\setminus\left\{ -1,+1 \right\}$ cannot be closed.

  Since $-1$ and $+1$ are the only integers that are not multiples of primes,
  \[
  \mathbb{Z}\setminus\left\{ -1,+1 \right\}=\bigcup\limits_{p\text{ prime}}S(0,p)
  .
  \]
  If the primes are finite, then from the second property of \cref{lem:2prop} we know that the right hand side is the finite union of closed sets, so it is also closed. Thus, $\mathbb{Z}\setminus\left\{ -1,+1 \right\}$ is closed.

  Therefore, we would reach a contradiction if there were only finitely many primes. Hence, the primes are infinite.
\end{proof}<++>

\section{Analytical Proofs}<++>

\begin{thebibliography}{12}
  \bibitem{<+bibkey+>} <++>
    \textit{Title}, Author, Place: Publisher Year
  \bibitem{bib:mathHistory}
    \textit{Mathematics and Its History}, Stillwell, Springer (2010)
  \bibitem{bib:proofsFromTheBook}
    \textit{Proofs from The Book}, Aigner, Ziegler, Springer-Verlag (1998)
  \bibitem{bib:Furstenberg}
    \textit{On the Infinitude of primes}, Furstenberg, American Mathematical Monthly (1955) 62 (5): 353
\end{thebibliography}<++>
\end{document}
